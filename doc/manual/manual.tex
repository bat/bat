\documentclass[11pt, a4paper]{article}

\usepackage{epsfig}

\setlength{\oddsidemargin}{0cm}
\setlength{\evensidemargin}{0cm}
\setlength{\topmargin}{-1cm}
\setlength{\textheight}{23cm}
\setlength{\textwidth}{16cm}

\pagestyle{headings}

\newcommand{\BC}{{\sc BayesianCalculator}}

%--------------------------------------------------------

\begin{document}

% --------------------------------------------------------
% title
% --------------------------------------------------------

\thispagestyle{empty} 

\title{\BC\ - A software package for Bayesian inference} 

\author{A.~Caldwell, D.~Kollar, K.~Kr\"oninger} 

\maketitle

\begin{abstract} 
Abstract. 
\end{abstract} 

\pagebreak 

% --------------------------------------------------------
% introduction
% --------------------------------------------------------

\section{Introduction}

The \BC\ software package is a framework for Bayesian inference. It
allows the user to perform an analysis from model comparisons,
parameter estimate to posterior model checking. The software is
written in C++ and provided in form of a library. It's main features
are (1)~it's flexibility to different types of applications (see
Section~\ref{subsection:problems}) and (2)~the modular structure of
the provided tools which allows the user to use different ready-made
code or provide own code. 

\subsection{Bayesian inference} 

\begin{itemize} 
\item General stuff about Bayesian inference. 
\end{itemize} 

\subsection{Addressed problems} 
\label{subsection:problems} 

The outcome of experiments in particle physics and cosmology are often
parameters extracted from a measured set of data. Different
statistical tools are developed for the variety of experiments and
measurements. The \BC\ software package was written with the intention
to provide a collection of tools for Bayesian analyses. \\ 

Three examples of applications in the realm of physics and cosmology
are part of the package. These are 

\begin{itemize}
% 
\item Neutrinoless double beta-decay. In experiments built to search
for rare processes few signal events have to be distinguished from a
non-negligible background. Common approximations, e.g. Gaussian
distributions of count rates, do not apply in this case. The
comparison of models, i.e., signal and background vs. background only,
are emphasized for such experiments. 
% 
\item Cosmological models.  
%
\item Structure functions. 
%
\end{itemize}

% --------------------------------------------------------
% analysis 
% --------------------------------------------------------

\section{Analysis chain}

\subsection{The whole chain} 

\begin{itemize} 
\item Model comparison 
\item Parameter estimate 
\item Goodness-of-fit test 
\end{itemize}

\subsection{Pure parameter estimate} 

% --------------------------------------------------------
% Code structure 
% --------------------------------------------------------

\section{Code structure}

% --------------------------------------------------------
% Examples
% --------------------------------------------------------

\section{Examples}

\subsection{Example 01: } 

\subsection{Example 02: Polynomial fitting} 

\subsection{Example 03: Background only?} 

% --------------------------------------------------------

\end{document} 


